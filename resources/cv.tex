% resume.tex
% vim:set ft=tex spell:

\documentclass[10pt,letterpaper]{article}
\usepackage[letterpaper,margin=0.75in]{geometry}
\usepackage[utf8]{inputenc}
\usepackage{ifthen}
\usepackage{mdwlist}
\usepackage[T1]{fontenc}
\usepackage{textcomp}
\usepackage{tgpagella}
\usepackage{hyperref}
\pagestyle{empty}
\setlength{\tabcolsep}{0em}

% indentsection style, used for sections that aren't already in lists
% that need indentation to the level of all text in the document
\newenvironment{indentsection}[1]%
{\begin{list}{}%
	{\setlength{\leftmargin}{#1}}%
	\item[]%
}
{\end{list}}

% opposite of above; bump a section back toward the left margin
\newenvironment{unindentsection}[1]%
{\begin{list}{}%
	{\setlength{\leftmargin}{-0.5#1}}%
	\item[]%
}
{\end{list}}

% format two pieces of text, one left aligned and one right aligned
\newcommand{\headerrow}[2]
{\begin{tabular*}{\linewidth}{l@{\extracolsep{\fill}}r}
	#1 &
	#2 \\
\end{tabular*}}

% make "C++" look pretty when used in text by touching up the plus signs
\newcommand{\CPP}
{C\nolinebreak[4]\hspace{-.05em}\raisebox{.22ex}{\footnotesize\bf ++}}

% Link Formatting
\hypersetup{
    colorlinks=true,
    linkcolor=blue,
    filecolor=magenta,
    urlcolor=cyan,
    pdftitle={Matthew J. Davis Resume/CV},
    pdfpagemode=FullScreen,
}

% add a header:
    % 1 - Name
    % 2 - Street Address
    % 3 - City & State
    % 4 - Zip
    % 5 - Phone Number
    % 6 - Email
    % 7 - Github
    % 8 - Website
\newcommand{\header}[8]{
    \filbreak
    \begin{center}
        {\LARGE \textbf{#1}}
        
        #2\ \ \textbullet
        \ \ #3\ \ \textbullet
        \ \ #4
        \\
        #5\ \ \textbullet
        \ \ #6
        \\
        #7\ \ \textbullet
        \ \ #8
    \end{center}

    \hrule
}

% define a work experience:
    % 1 - Company Name
    % 2 - Title
    % 3 - Start Date
    % 4 - End Date
    % 5 - Supervisor
    % 6 - Description
\newcommand{\experience}[6]{
    \filbreak
    \item
	\headerrow
		{\textbf{#1}}
            %\ifthenelse{\equal{#4}{}}
                {{\textbf{\emph{#3}}}}
            % {{\textbf{\emph{#3 -- #4}}}}
	\\
	\headerrow
		{\emph{#2}}
            {\emph{Supervisor: #5}}
	#6
}

% define an education experience:
    % 1 - College/University Name
    % 2 - Year
    % 3 - Degree
    % 4 - Advisor
    % 5 - Accolades
\newcommand{\education}[5]{
    \filbreak
    \item
	\headerrow
		{\textbf{#1}}
		{\textbf{\emph{#2}}}
	\\
	\headerrow
		{\emph{#3}}
		{\emph{#4}}
	#5
}

% define a new research project:
    % 1 - Title
    % 2 - Date
    % 3 - Advisor
    % 4 - Collaborators
    % 5 - Description
\newcommand{\research}[5]{
    \filbreak
    \item
        \headerrow
            {\textbf{#1}}
            {\textbf{\emph{#2}}}
        \\
        \headerrow
            {\emph{Advisor: #3}}
            {\emph{Collaborators: #4}}
        #5
}

% define a new publication:
\newcommand{\publication}[3]{
    \filbreak
    \item[\textbf{#1}]- #2 \href{#3}{Link}
}

% Content
\begin{document}

\header
    {Matthew J. Davis}
    {3918 Teal Fern Ct.} {Houston, TX.} {77059}
    {(713)-806-0852}
    {\href{mailto:matthewdavis.professional@gmail.com} {matthewdavis.professional@gmail.com}}{\url{https://github.com/davis-matthew}}
    {\url{https://davis-matthew.github.io}}

\vspace{-0.4em}
\subsection*{Education}
\begin{itemize}
	\parskip=0.1em

        \education
            {Georgia Institute of Technology} {2022 - Present} 
            {Ph.D Computer Science} {Advisors: Dr. Vivek Sarkar, Dr. Vijay Ganesh}
            {
                \begin{itemize*}
                    \item President's Fellow
                \end{itemize*}
            }

	\education
            {Texas A\&M University -- College Station} {2022} 
            {B.S. of Computer Science \& Engineering} {}
            {
                \begin{itemize*}
                    \item Engineering Honors
                    \item Summa Cum Laude
                    \item Undergraduate Research Scholar
                \end{itemize*}
            }

\end{itemize}

\hrule
\vspace{-0.4em}
\subsection*{Technical Skills}
\begin{indentsection}{\parindent}
\hyphenpenalty=1000
\begin{description*}
	\item[Languages:]
        Java, \CPP, Python, Cuda, C, Bash, SQL, JavaScript 
	\item[Tools \& Frameworks:]
	MPI, OpenMP, Thread Sanitizer, LLVM
\end{description*}
\end{indentsection}

\hrule
\vspace{-0.4em}
\subsection*{Experience}
\begin{itemize}
	\parskip=0.1em

    \experience
            {Oak Ridge National Lab}
            {Research Student Intern}
            {2025} {}
            {Dr. Keita Teranishi}
            {
                \begin{itemize*}
                    \item Developed a novel legacy code-translation infrastructure 
    
                    \item Work under review for publication
                \end{itemize*}               
            }

	\experience
            {Helios Solutions}
            {Software Engineering Intern}
            {2022} {}
            {Mr. Joel Busa}
            {
                \begin{itemize*}
                    \item Developed software and developer infrastructure tools used by customer Intuitive Machines on their lunar landers: IM-1, IM-2, \& IM-3. 
    
                    \item Created graphic user interface tools for customer TTTech's switch and cable modeling.
                \end{itemize*}               
            }

	\experience
            {Argonne National Lab}
            {Research Aide}
            {2021} {}
            {Dr. Yanfei Guo}
            {
                \begin{itemize*}
                    \item Assisted the pmodel's MPICH team by integrating automated concurrency bug detection passes into their CI systems.
                    \item Adapted symbolic execution tool KLEE to automatically generate values for unit testing of MPI library functions.
                \end{itemize*}
            }

\end{itemize}

\hrule
\vspace{-0.4em}
\subsection*{Research}
\begin{itemize}
        \parskip=0.1em

        \research
        {Verified LLM-Based Code Translation} {2024 -- Present}
        {Dr. Vijay Ganesh} {}
        {
            \begin{itemize*}
                \item Translating legacy \& non-portable HPC code to modern languages and programming models
                \item Translating Online Encyclopedia of Integer Sequences (OEIS) entries to Lean and C/C++
                \item Translating loops to loop-invariant annotated loops for program verification
                \item Large Language Models in a translation loop with validators providing feedback to guide repair
            \end{itemize*}
        }

        \research
        {Configuration Generation for NN Verification} {2024 -- Present}
        {Dr. Vijay Ganesh} {Salil Kamath}
        {
            \begin{itemize*}
                \item Created a reinforcement learning model trained using fuzzed NN instances to generate per-instance hyperparameter configurations for NN verifiers
                \item Optimized solve time, keeping up with expert-tuned configurations.
            \end{itemize*}
        }

        \research
        {Hybrid OpenMP Data Mapping Violation Detection \& Repair} {2022 -- Present}
        {Dr. Vivek Sarkar} {Dr. Lechen Yu}
        {
            \begin{itemize*}
                \item Reducing Arbalest Instrumentation using compiler analysis
                \item Further optimizing by selecting a subset of target regions using OMPSanitizer's static analysis results
                \item Optimizing and repairing mapping data movement via OMPMemOpt \& Arbalest results
            \end{itemize*}
        }

        \research
        {HPCTest - Detecting Heterogeneous Bugs in Scientific Computing Software} {2022 -- 2024}
        {Dr. Vivek Sarkar} {Manish Motwani}
        {
            \begin{itemize*}
                \item Combined LLM input generation, static analysis, guided fuzzing, \& differential testing to create a fuzzing-based bug detection system which is scalable to large HPC \& Scientific Computing systems.
                \item Developed tools to guide the fuzzer using feedback based off analysis of runtime values and execution patterns.
            \end{itemize*}
        }

        \research
        {Extending OpenRace for CUDA Race Detection} {2020 -- 2021}
        {Dr. Jeff Huang} {Brad Swain, Coderrect Inc.}
        {
            \begin{itemize*}
                \item Extended static data race detection tool OpenRace to model and detect races in CUDA 8 and before (no cooperative groups) and fixed flaws in the OpenMP Device offload modeling which improved results on the DataRaceBench benchmark.
                \item This work was merged into the OpenRace repository.
            \end{itemize*}
        }
        
        \research
        {Dynamatic OpenMP Race Detector} {2019 -- 2020}
        {Dr. Jeff Huang} {Dylan Theriot, Fatma Elsheimy}
        {
            \begin{itemize*}
                \item Developed a hybrid (static \& dynamic) program analysis tool. This tool finds data race bugs in OpenMP programs by combining results from the HPCRace static analysis tool \& Google Thread Sanitizer reports. 
                \item Improved the performance on benchmark DataRaceBench, keeping all true positives of HPCRace and disproving all false positives.
                \item This work is published at: \href{https://hdl.handle.net/1969.1/194411}{Dynamatic: An OpenMP Race Detection Tool Combining Static and Dynamic Analysis}
            \end{itemize*}
        }
        
        \research
        {NEO-UFO} {2019}
        {Dr. Jeff Huang} {Yahui Sun, Matthew Chen, Andrew Chin, Andreas Tsouloupas}
        {
            \begin{itemize*}
                \item Wrote a static analysis pass to identify regions in the Chromium browser base which were unlikely to have Use-After-Free (UAF) bugs. Converted these regions into Thread Sanitizer blacklist files to toggle off the expensive tracing and analysis for dynamic analysis tool UFO, greatly reducing the overhead.
            \end{itemize*}
        }

\end{itemize}

\hrule
\vspace{-0.4em}
\subsection*{Honors \& Awards}
\begin{itemize}
        \parskip=0.1em
        
        \item Eagle Scout

\end{itemize}

\hrule
\vspace{-0.4em}
\subsection*{Publications}
\begin{itemize}
        \parskip=0.1em
        \item[]
            % \begin{center}
            %    \textbf{Georgia Institute of Technology (2022 -- Present)}
            %\end{center}
            %\begin{center}
            %    \textbf{Texas A\&M University (2018 -- 2022)}
            %\end{center}
            \begin{itemize*}
                \publication
                    {2025} 
                    {Salil Kamath, \textbf{Davis, Matthew James}, Jonathan Andreasen, Yatis Dodia, and Vijay Ganesh. Automated VNN solver configuration selection via deep reinforcement learning. In \textit{International Symposium on AI Verification}.}
                    {https://openreview.net/forum?id=PT9TETWLad}
            \end{itemize*}
            \begin{itemize*}
                \publication
                    {2022} 
                    {\textbf{Davis, Matthew James}; Theriot, Dylan (2022). Dynamatic: An OpenMP Race Detection Tool Combining Static and Dynamic Analysis. Bachelor's Thesis.}
                    {https://hdl.handle.net/1969.1/194411}
            \end{itemize*}
\end{itemize}


\end{document}
